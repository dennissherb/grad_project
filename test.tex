%\title{יומן פיתוח פרוייקט}
\documentclass{article}
\usepackage[utf8x]{inputenc}
\usepackage[english,hebrew]{babel}
\usepackage{datetime}
\usepackage{ragged2e}
\selectlanguage{hebrew}
\usepackage[top=2cm,bottom=2cm,left=2.5cm,right=2cm]{geometry}
\title{יומן פיתוח פרוייקט}
\author{דניס שרבקוב}
\newdate{Initialcreation}{01}{09}{2023}
\newdate{firstchange}{14}{02}{2024}
\date{\displaydate{Initialcreation}}

\begin{document}
\maketitle

\tableofcontents

\newpage
\section{הפרוייקט}
מטרת הפרוייקט היא לספק אתר שישמש ככלי עזר לקבלת החלטות לגבי קניות של מוצרי הבית היום-יומיים האתר יכיל מגוון רחב של מידע לגבי שלל מוצרים ויספק אלטרנטיבות קניה "ירוקות" למשתמש על מנת שיוכל לבצע קניות באופן ששומר על הסביבה ברמה המירבית
\subsection{כיצד עובד האתר?}
האתר ימלא את תפקידו בעזרת שימוש במערכת דמויית \L{Wiki} שתאפשר למשתמשים להזין מידע לאתר בכוחות עצמם לגבי מגוון מוצרים
\newpage

\section{יומן שינויים}
זהו יומן השינויים שבו אני אתאר וארשום את השינויים שאני עושה לפרוייקט, בנוסף לכך ארשום גם כן את דרך התכנון והחשיבה לגבי התכנות והיישום של הפרוייקט 

\subsection{\protect\displaydate{firstchange}}

שינויים שערכתי היום עוסקים באימפלמנטציה של עדכון משתמש. שינוי חשוב הוא ההוספה של שימוש ב-\L{ID} לפעולת העדכון כדי שיהיה קל יותר לזהות את השורה בטבלת המשתמשים שבה רוצים לעשות שינוי, בנוסף לדרך הזיהוי הנוכחית שנעזרת ב-\L{UQ} כמו אימייל ושם משתמש. שינוי נוסף שיש לו חשיבות גבוהה הוא הוספת \L{Try \& Catch} לפעולת ה-\L{API} כדי למנוע קריסה של התוכנה או החזר לא וורבאלי של שגיאה. שינוי נוסף ל-\L{API} הוא שכעת פעולת העדכון תחזיר את המשתמש החדש שנמצא בטבלה לאחר העדכון על מנת שהעצם שמייצג את המשתמש תמיד יהיה עדכני וזהה לזה שקיים בטבלה ללא צורך בהפעלה נוספת של ה-\L{API}


\end{document}